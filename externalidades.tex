\section{Externalidades}

\subparagraph{Bibliografía:} \cite{binhoff} 
	
\emph{Definición:} Existe una externalidad cuando el bienestar de
un consumidor o las posibilidades de producción de una firma está
directamente afectadas por las acciones de otro agente en la
economía.

\subsection{Externalidades en la producción}

En esta sección se sigue el desarrollo de \citet{binhoff}.

Supuestos:

\begin{itemize}
 \item 2 firmas, $x$ e $y$ producen 2 bienes.
 \item Un factor de producción L
 \item La firma $x$ es la que genera la externalidad
 (contaminación):
$$\frac{\partial y}{\partial x}<0$$
 \item Funciones de producción
    \begin{eqnarray*}
        y&=&y(L_{y},x) \\
        x&=&x(L_{x})
    \end{eqnarray*}
\end{itemize}

Un razonamiento \emph{naif} sería plantear producción cero para
lograr contaminación cero. Pero, como vamos a demostrar \emph{la
asignación eficiente no implica contaminación cero}

Supongamos que $x$ e $y$ se fusionan, y son manejadas por un \textit{manager}.
Por lo tanto, se va a tener en cuenta que el aumento de la
producción de $x$ implica la caída de $y$. Cuando se tiene en cuenta este punto, \emph{se internaliza la
externalidad.}

$$ Max \: \pi = P_x \, x(L_x)+P_y \, y(L_y,x(L_x))-w(L_x+L_y)$$

con $P_x, P_y$ y $w$ dados.\\


\emph{Condiciones de Primer Orden:}
\begin{eqnarray*}
	\frac{\partial \pi}{\partial L_x} &:& P_x.MPL_x+P_y \frac{\partial
	y}{\partial x}MPL_x-w=0 \\
	\frac{\partial \pi}{\partial L_y}&:& P_y.MPL_y-w=0
\end{eqnarray*}

Entonces:

\begin{eqnarray}
 w &=& P_x .MPL_x + P_y.MPL_y \frac{\partial y}{\partial x}-w=0 \label{px1}\\
 w &=& P_y.MPL_y \label{py1}
\end{eqnarray}

Las 2 firmas fusionadas eligen $L$ tal que:
\begin{itemize}
 \item en $y$, $w=PML_y$
 \item en $x$, $w=PML_x+\mbox{Daño marginal en }y$
\end{itemize}

Despejando los precios de los 2 bienes de las ecuaciones (\ref{py1}) y (\ref{px1}):

\begin{eqnarray}
 P_x &=& \frac{w}{PML_x}-P_y \frac{\partial y}{\partial x} \n\\
 P_y &=& \frac{w}{MPL_y} \label{optimo}
\end{eqnarray}

La tasa marginal de sustitución sería:

$$\frac{P_x}{P_y}=\frac{\frac{w}{PML_x}-P_y \frac{\partial
y}{\partial x}}{\frac{w}{MPL_y}}$$

El precio relativo es mayor que el obtenido en el la situación en que la externalidad no se internaliza.
Si las firmas no se fusionan, $P_x$ no reflejaría la externalidad que genera $x$ sobre $y$.

Si no existen externalidades, la $FPP$ se mueve hacia adentro (Figura \ref{fig:grafico1}).


\begin{figure}
	\centering
	\caption{$FPP$ con Externalidades}
	%%Created by jPicEdt 1.x
%Standard LaTeX format (emulated lines)
%Fri Aug 05 11:42:53 ART 2005
\unitlength 1mm
\begin{picture}(105.00,90.00)(0,0)

\linethickness{0.15mm}
%Polygon 1 0(10.00,90.00)(10.00,10.00) 
\put(10.00,10.00){\line(0,1){80.00}}
\put(10.00,90.00){\vector(0,1){0.12}}
%End Polygon

\linethickness{0.15mm}
%Polygon 0 1(10.00,10.00)(105.00,10.00) 
\put(10.00,10.00){\line(1,0){95.00}}
\put(105.00,10.00){\vector(1,0){0.12}}
%End Polygon

\linethickness{0.15mm}
%Bezier 0 0(10.00,80.00)(90.00,85.00)(95.00,10.00)
\qbezier(10.00,80.00)(90.00,85.00)(95.00,10.00)
%End Bezier

\linethickness{0.15mm}
%Bezier 0 0(10.00,80.00)(70.00,65.00)(95.00,10.00)
\qbezier(10.00,80.00)(70.00,65.00)(95.00,10.00)
%End Bezier

\linethickness{0.15mm}
%Polygon 1 0(70.00,65.00)(70.00,10.00) dash=1.00
\multiput(70.00,10.00)(0,2.00){28}{\line(0,1){1.00}}
\put(70.00,65.00){\vector(0,1){0.12}}
%End Polygon

\linethickness{0.15mm}
%Polygon 1 0(70.00,65.00)(10.00,65.00) dash=1.00
\multiput(10.00,65.00)(1.97,0){31}{\line(1,0){0.98}}
\put(70.00,65.00){\vector(1,0){0.12}}
%End Polygon

\linethickness{0.15mm}
%Polygon 1 0(70.00,46.71)(10.00,46.71) dash=1.00
\multiput(10.00,46.71)(1.97,0){31}{\line(1,0){0.98}}
\put(70.00,46.71){\vector(1,0){0.12}}
%End Polygon

\put(60.00,10.00){\makebox(0,0)[cc]{}}

\put(60.00,10.00){\makebox(0,0)[cc]{}}

\put(60.00,10.00){\makebox(0,0)[cc]{}}

\put(70.13,7.11){\makebox(0,0)[bc]{$x_0$}}

\put(104.87,8.16){\makebox(0,0)[cc]{$x$}}

\put(65.00,5.00){\makebox(0,0)[cc]{}}

\put(7.76,89.34){\makebox(0,0)[cc]{$y$}}

\put(7.76,64.87){\makebox(0,0)[cc]{$y_0$}}

\put(100.00,50.00){\makebox(0,0)[cc]{}}

\put(5.00,80.00){\makebox(0,0)[cc]{}}

\put(69.47,7.50){\makebox(0,0)[cc]{}}

\put(70.00,0.00){\makebox(0,0)[cc]{}}

\put(7.3,46.32){\makebox(0,0)[cc]{$y_0^{ext}$}}

\put(6.45,46.97){\makebox(0,0)[cc]{}}

\linethickness{0.15mm}
%Polygon 0 1(54.34,60.26)(74.34,78.82) 
\multiput(54.34,60.26)(0.13,0.12){155}{\line(1,0){0.13}}
\put(74.34,78.82){\vector(1,1){0.12}}
%End Polygon

\put(93.68,81.18){\makebox(0,0)[cc]{FPP con Externalidades}}

\put(85.53,82.63){\makebox(0,0)[cc]{}}

\put(72.63,82.24){\makebox(0,0)[cc]{}}

\put(78.55,77.11){\makebox(0,0)[cc]{}}

\put(94.34,78.03){\makebox(0,0)[cc]{Internalizadas}}

\put(78.16,75.39){\makebox(0,0)[cc]{}}

\put(78.03,76.84){\makebox(0,0)[cc]{}}

\linethickness{0.15mm}
%Polygon 0 1(75.53,40.92)(85.53,59.08) 
\multiput(75.53,40.92)(0.12,0.22){83}{\line(0,1){0.22}}
\put(85.53,59.08){\vector(1,2){0.12}}
%End Polygon

\put(92.37,61.05){\makebox(0,0)[cc]{FPP sin }}

\put(83.82,62.63){\makebox(0,0)[cc]{}}

\put(97.89,58.55){\makebox(0,0)[cc]{externalidades}}

\put(87.50,56.84){\makebox(0,0)[cc]{}}

\end{picture}
 
	\label{fig:grafico1}
\end{figure}

\begin{figure}
	\centering
	\caption{Ganancias en el Bienestar}
	%%Created by jPicEdt 1.x
%Standard LaTeX format (emulated lines)
%Fri Aug 05 12:25:37 ART 2005
\unitlength 1mm
\begin{picture}(109.08,90.00)(0,0)

\linethickness{0.15mm}
%Polygon 1 0(10.00,90.00)(10.00,10.00) 
\put(10.00,10.00){\line(0,1){80.00}}
\put(10.00,90.00){\vector(0,1){0.12}}
%End Polygon

\linethickness{0.15mm}
%Polygon 0 1(10.00,10.00)(105.00,10.00) 
\put(10.00,10.00){\line(1,0){95.00}}
\put(105.00,10.00){\vector(1,0){0.12}}
%End Polygon

\linethickness{0.15mm}
%Bezier 0 0(10.00,80.00)(70.00,65.00)(95.00,10.00)
\qbezier(10.00,80.00)(70.00,65.00)(95.00,10.00)
%End Bezier

\put(60.00,10.00){\makebox(0,0)[cc]{}}

\put(60.00,10.00){\makebox(0,0)[cc]{}}

\put(60.00,10.00){\makebox(0,0)[cc]{}}

\put(104.87,8.16){\makebox(0,0)[cc]{$x$}}

\put(65.00,5.00){\makebox(0,0)[cc]{}}

\put(7.76,89.34){\makebox(0,0)[cc]{$y$}}

\put(100.00,50.00){\makebox(0,0)[cc]{}}

\put(5.00,80.00){\makebox(0,0)[cc]{}}

\put(69.47,7.50){\makebox(0,0)[cc]{}}

\put(70.00,0.00){\makebox(0,0)[cc]{}}

\put(6.45,46.97){\makebox(0,0)[cc]{}}

\put(93.68,81.18){\makebox(0,0)[cc]{FPP con Externalidades}}

\put(85.53,82.63){\makebox(0,0)[cc]{}}

\put(72.63,82.24){\makebox(0,0)[cc]{}}

\put(78.55,77.11){\makebox(0,0)[cc]{}}

\put(94.34,78.03){\makebox(0,0)[cc]{Internalizadas}}

\put(78.16,75.39){\makebox(0,0)[cc]{}}

\put(78.03,76.84){\makebox(0,0)[cc]{}}

\put(83.82,62.63){\makebox(0,0)[cc]{}}

\put(87.50,56.84){\makebox(0,0)[cc]{}}

\linethickness{0.15mm}
%Bezier 0 0(43.95,82.24)(58.16,44.74)(100.92,34.47)
\qbezier(43.95,82.24)(58.16,44.74)(100.92,34.47)
%End Bezier

\linethickness{0.15mm}
%Bezier 0 0(29.61,83.03)(38.95,35.53)(99.08,21.97)
\qbezier(29.61,83.03)(38.95,35.53)(99.08,21.97)
%End Bezier

\linethickness{0.15mm}
%Polygon 0 1(52.89,50.66)(35.66,38.16) dash=1.00
\multiput(35.66,38.16)(1.64,1.19){11}{\multiput(0,0)(0.16,0.12){5}{\line(1,0){0.16}}}
\put(35.66,38.16){\vector(-4,-3){0.12}}
%End Polygon

\put(35.53,36.71){\makebox(0,0)[cc]{Ganancias de la }}

\put(35.00,33.95){\makebox(0,0)[cc]{Internalización}}
\linethickness{0.15mm}
%Polygon 0 0(96.05,26.58)(33.82,77.24) 
\multiput(33.82,77.24)(0.15,-0.12){422}{\line(1,0){0.15}}
%End Polygon

\put(106.84,30.39){\makebox(0,0)[cc]{Ext. Internalizada}}

\linethickness{0.15mm}
%Polygon 0 0(95.53,22.24)(61.84,32.37) 
\multiput(61.84,32.37)(0.40,-0.12){84}{\line(1,0){0.40}}
%End Polygon

\put(109.08,19.48){\makebox(0,0)[cc]{Ext. sin internalizar}}

\put(86.58,22.90){\makebox(0,0)[cc]{$A$}}

\end{picture}

	\label{fig:grafico2}
\end{figure}

Si la externalidad no es internalizada, la maximización del consumidor lo situaría en un punto como $A$, pero allí $TMS \neq TMT$. Por lo tanto, el individuo se está perdiendo de aumentar su bienestar (área sombreada). Al internalizarla, se situa en el punto $B$, donde aumenta la relación de precios relativos $\frac{P_x}{P_y}$. Es un último en el sentido de Pareto.

La Figura \ref{fig:grafico3} muestra, de manera grafica, un mercado con una externalidad negativa, en el esquema de un modelo de Equilibrio Parcial.

\begin {figure}
\caption{Equilibrio Parcial}
	\centering
	%%Created by jPicEdt 1.x
%Standard LaTeX format (emulated lines)
%Fri Aug 05 12:14:45 ART 2005
\unitlength 1mm
\begin{picture}(105.00,90.00)(0,0)

\linethickness{0.15mm}
%Polygon 1 0(10.00,90.00)(10.00,10.00) 
\put(10.00,10.00){\line(0,1){80.00}}
\put(10.00,90.00){\vector(0,1){0.12}}
%End Polygon

\linethickness{0.15mm}
%Polygon 0 1(10.00,10.00)(105.00,10.00) 
\put(10.00,10.00){\line(1,0){95.00}}
\put(105.00,10.00){\vector(1,0){0.12}}
%End Polygon

\put(60.00,10.00){\makebox(0,0)[cc]{}}

\put(60.00,10.00){\makebox(0,0)[cc]{}}

\put(60.00,10.00){\makebox(0,0)[cc]{}}

\put(104.87,7.79){\makebox(0,0)[cc]{$x$}}

\put(65.00,5.00){\makebox(0,0)[cc]{}}

\put(7.6,89.34){\makebox(0,0)[cc]{$P_x$}}

\put(100.00,50.00){\makebox(0,0)[cc]{}}

\put(5.00,80.00){\makebox(0,0)[cc]{}}

\put(69.47,7.50){\makebox(0,0)[cc]{}}

\put(70.00,0.00){\makebox(0,0)[cc]{}}

\put(6.45,46.97){\makebox(0,0)[cc]{}}

\put(85.53,82.63){\makebox(0,0)[cc]{}}

\put(72.63,82.24){\makebox(0,0)[cc]{}}

\put(78.55,77.11){\makebox(0,0)[cc]{}}

\put(78.16,75.39){\makebox(0,0)[cc]{}}

\put(78.03,76.84){\makebox(0,0)[cc]{}}

\put(83.82,62.63){\makebox(0,0)[cc]{}}

\put(87.50,56.84){\makebox(0,0)[cc]{}}

\linethickness{0.15mm}
%Polygon 0 0(21.32,25.26)(72.63,81.58) 
\multiput(21.32,25.26)(0.12,0.13){428}{\line(0,1){0.13}}
%End Polygon

\linethickness{0.15mm}
%Polygon 0 0(29.47,19.47)(96.05,63.16) 
\multiput(29.47,19.47)(0.18,0.12){364}{\line(1,0){0.18}}
%End Polygon

\linethickness{0.15mm}
%Polygon 0 0(20.53,77.37)(88.68,23.42) 
\multiput(20.53,77.37)(0.15,-0.12){450}{\line(1,0){0.15}}
%End Polygon

\linethickness{0.15mm}
%Polygon 0 0(48.42,55.26)(48.42,10.00) dash=1.00
\multiput(48.42,10.00)(0,2.01){23}{\line(0,1){1.01}}
%End Polygon

\linethickness{0.15mm}
%Polygon 0 0(64.61,42.50)(64.61,10.00) dash=1.00
\multiput(64.61,10.00)(0,1.97){17}{\line(0,1){0.98}}
%End Polygon

\linethickness{0.15mm}
%Polygon 0 0(48.42,55.39)(10.00,55.39) dash=1.00
\multiput(10.00,55.39)(1.97,0){20}{\line(1,0){0.99}}
%End Polygon

\linethickness{0.15mm}
%Polygon 0 0(64.47,42.50)(10.00,42.50) dash=1.00
\multiput(10.00,42.50)(1.98,0){28}{\line(1,0){0.99}}
%End Polygon

\put(63.29,7.76){\makebox(0,0)[cc]{\overline{x}}}

\put(47.37,7.76){\makebox(0,0)[cc]{$x^*$}}

\put(5.53,55.40){\makebox(0,0)[cc]{$P_x^*$}}

\put(4.21,42.50){\makebox(0,0)[cc]{\overline{$P_x$}}}

\put(76.18,82.24){\makebox(0,0)[cc]{CMg Social}}

\put(95.79,67.24){\makebox(0,0)[cc]{Cmg Privado}}

\end{picture}
 
	\label{fig:grafico3}
\end{figure}

El equilibrio social es $(P^*_x, x^*)$ y el privado $(\overline{P^*_x}, \overline{x^*})$. El equilibrio competitivo genera una producción menor a la socialmente óptima.


\subsection {Externalidades en el Consumo}

\begin{itemize}
 \item Un consumidor
 \item Dos bienes, $x$ e $y$
 \item Un factor: $L$
 \item $x$ produce humo ($s$), es el que genera la externalidad.
\end{itemize}

\begin{eqnarray*}
 U &=& U(x,y;s) \\
 x &=& x(L_x) \\
 y &=& y(L_y) \\
 s &=& s(x) \; \mbox{con} \; \frac{\partial s}{\partial x}>0 \\
 L_x+L_y &=& \overline{L}
\end{eqnarray*}

Con $U_x>0; U_y>0; U_s<0$\\

El problema que resuelve la empresa fusionada es:

$$Max \; U(x(L_x);y(\overline{L}-L_x);s(x(L_x)))$$

\emph{Condiciones de Primer Orden:}:

$$\frac{\partial U}{\partial{L_x}}=U_x \frac{\partial
x}{\partial{L_x}}-U_y\frac{\partial y}{\partial{L_y}}+U_s
\frac{\partial s}{\partial{x}} \frac{\partial x}{\partial{L}} $$

Reordenando:

$$ MPL_x(U_x+U_s \frac{\partial s}{\partial{x}})=MPL_y.U_y$$

Entonces:

\begin{eqnarray*}
	TMT_{yx} &=& \frac{MPL_y}{MPL_x} = \frac{U_x+U_s.\frac{\partial
s}{\partial x}{U_y}}{U_y} \\
	TMT_{yx} &=& MRS_{yx}+MRS_{ys}.\frac{\partial
 s}{\partial{x}}=MPL_y.U_y \\
	TMT_{yx} &=& \frac{U_x}{U_y}+\frac{U_s}{U_y}\frac{\partial s}{\partial y}
\end{eqnarray*}

donde $\frac{U_s}{U_y}\frac{\partial s}{\partial y}$ es el daño
marginal. \\

Por lo tanto:

$$TMT_{yx} = MRS_{yx}+MRS_{ys}.\frac{\partial
 s}{\partial{x}}=MPL_y.U_y$$

Entonces:
\begin{eqnarray}
	MRS_{xy}&>&TMT{xy} \label{eqgrafico1}\\
	\frac{P_x}{P_y}&=&MRS_{xy}=TMT_{xy}-MRS_{ys}\frac{\partial
s}{\partial{x}} \n
\end{eqnarray}

La ecuación \ref{eqgrafico1} representa el punto $A$ en la Figura \ref{fig:grafico1}

El precio relativo de $x$ va a ser mas alto que cuando no se
internaliza la externalidad.

\subsection{Mecanismos para solucionar las fallas de mercado}

\subsubsection{Impuesto Pigouviano}
Una de las posibles soluciones a los problemas generados por la externalidad
sería establecer un impuesto de monto similar al daño marginal que produce la
externalidad en el nivel último de producción o consumo. \\
En el caso de una externalidad en la producción, el impuesto último
sería:

$$t_x=-P_y\frac{\partial y}{\partial x}(x^*)$$

\paragraph{Demostración}

La firma $x$:

$$ Max \; \pi = P_x.x-c(x)-t_x.x$$

\emph{Condiciones de Primer Orden:}:

\begin{eqnarray}
 P_x-c'(x)-t_x &=& 0 \n \\
 P_x &=&c'(x)+t_x \n \\
 P_x &=&c'(x)-P_y \frac{\partial y}{\partial x}(x^*) \label{optimo2}
\end{eqnarray}
$P_x$ es igual al que obtuvimos en \ref{optimo}, cuando ambas firmas
se fusionaban.\\

\emph{Crítica:} Es necesario contar con información ex-ante sobre
cual es la magnitud del daño marginal social.

\subsubsection{Negociación}
Esta solución, desarrollada por \citet{coase}, critica el enfoque de
Pigou. Coase establece que si se cumplen ciertas condiciones no es
necesario una intervención en el mercado. Requiere que los derechos
de propiedad están establecidos de manera clara. Independientemente
de quien posea la propiedad, se arribará al mismo resultado eficiente.

\begin{enumerate}
\item $y$ posee los derechos a estar libre de contaminación
\label{casoi}

Si $x$ le paga a $y$, para que este le permita contaminar y
producir, se genera un \emph{mercado auxiliar} de derechos de
propiedad.

$$Max \; \pi_y=P_y. y(L_y,x)-wL_y+tx$$

$t$ es el precio que se fija para la contaminación por unidad de
producción.

Derivando con respecto a los argumentos $x$ y ·$L_y$:

\begin{eqnarray}
\frac{\partial \pi_y}{\partial x} &=& P_y \frac{\partial y}{\partial
x}+t=0 \n \\
t & = & -P_y \frac{\partial y}{\partial x} \label{tx1}
\end{eqnarray}

$$\frac{\partial\pi_y}{\partial L_y}=P_y\frac{\partial y}{\partial
L}-w=0$$

La empresa $x$ resuelve:

$$Max \; \pi_x=P_x. x(L_x)-wL_x-tx$$

\emph{Condiciones de Primer Orden:}:

\begin{eqnarray}
\frac{\partial \pi_x}{\partial L_x} &=& (P_x-t) \frac{\partial
x}{\partial L_x}+w=0 \n \\
t & = & -P_x - \frac{w}{MPL_x} \label{tx2}
\end{eqnarray}


Igualando (\ref{tx1}) y (\ref{tx2}), el precio de equilibrio que se obtiene es:

\begin{eqnarray}
t & = & -P_y \frac{\partial y}{\partial x} = -P_x -
\frac{w}{MPL_x}  \n \\
P_x&=&\frac{w}{MPL_x}-P_y \frac{\partial y}{\partial x}
\end{eqnarray}

Es idéntico al caso en que las firmas se fusionan.

\item $x$ el tiene derecho a contaminar.

$y$ debe pagarle a $x$ para tener derecho al aire libre. Suma $b$
por unidad de producción que $x$ disminuye su producción, a partir
de \~{a}

$$Max \; \pi_y=P_y. y(L_y,x)-wL_y-b(x-x)$$

\emph{Condiciones de Primer Orden:}:

\begin{eqnarray}
\frac{\partial \pi_y}{\partial x} &=& P_y \frac{\partial y}{\partial
x}+b=0 \n \\
b & = & -P_y \frac{\partial y}{\partial x} \label{b1}
\end{eqnarray}

$$\frac{\partial\pi_x}{\partial L_y}=P_y\frac{\partial y}{\partial
L}-w=0$$

La empresa $x$:

$$Max \; \pi_x=P_x.x(L_x)-wL_x+b(x-x(L_x))$$

\emph{Condiciones de Primer Orden:}:

\begin{eqnarray}
\frac{\partial \pi_x}{\partial L_x} &=& (P_x-b) \frac{\partial
x}{\partial L}-w=0 \n \\
b & = & P_x- \frac{w}{MPL_x} \label{b2}
\end{eqnarray}

Igualando (\ref{b1}) y (\ref{b2}):

\begin{eqnarray}
P_x- \frac{w}{MPL_x} &=& -P_y \frac{\partial y}{\partial x} \n \\
P_x &=& \frac{w}{MPL_x}-P_y \frac{\partial y}{\partial x}
\end{eqnarray}

El $P_x$ es igual al resultante al Caso 1 y al que las firmas se
fusionan.

\emph{Conclusión:} \marginpar{\emph{Teorema de Coase}} El Teorema de Coase establece que si existen
derechos de propiedad bien establecidos, que se puedan hacer cumplir
por leyes, y los costos de transacción son cero, la asignación
resultante va a ser pareto última, sin importar quien posee los
derechos de propiedad.
\end{enumerate}

La diferencia entre los Casos 1 y 2 es que en cada situación,
existirá una distribución del ingreso diferente.
